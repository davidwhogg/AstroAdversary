\documentclass[12pt, letterpaper]{article}

\begin{document}

\section*{Finding and understanding adversarial attacks against machine-learning
          methods in astrophysics}

\noindent
Villar, Hogg, others

\paragraph{abstract:}
% Context
In classification contexts, most machine-learning methods are
susceptible to adversarial attacks, in which obviously small or
uninformative changes to input data can be found that lead to substantial
changes in the output classifications of those data.
The great successes of these kinds of attacks undermines the idea that
these methods---which are becoming more prevalent---are valid for use
in scientific projects
% Aims
In this contribution, we compare different general classes of
machine-learning methods in terms of their susceptibility to
adversarial attacks.
% Methods
We generalize the concept of an adversarial attack to the context of
regressions, which is the context of most machine-learning
applications in astrophysics.
We search for attacks against some popular classes of models in
astronomy, using examples from stellar spectroscopy for demonstration
purposes (although we expect the results to be general).
% Results
We find, as expected, that the standard practices of regression with a
flexible machine-learning method (such as deep learning), even with
good training data, are generally susceptible to impressive
adversarial attacks.
We find that methods in which the machine-learning method is used to
make a generative model for the data (from a latent space or label
space) are far less susceptible to such attacks, probably because they
conform more closely to our physical knowledge.
Even methods that are susceptible to attacks can be made less
susceptible by modifications to the model structure or to the training
regime.
% Conclusion
We comment on the relationships between adversarial attacks,
generalizability, interpretability, and causal structure.

\end{document}
